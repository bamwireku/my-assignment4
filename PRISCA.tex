\documentclass[10pt,a4paper]{article}
\usepackage{cite}
\usepackage{zed-csp,graphicx,color}
\begin{document}
\begin{titlepage}
 \begin{figure}[h]
  \centerline{\small MAKERERE 
  \includegraphics[width=0.1\textwidth]{muklog} UNIVERSITY}
\end{figure}
\centerline{COLLEGE OF COMPUTING AND INFORMATIC SCIENCES}
\paragraph{•}
\centerline{DEPARTMENT OF COMPUTER SCIENCE\\}
\paragraph{•}

\centerline{COURSEWORK THREE: RESEARCH METHODOLOGY(BIT 2207)\\}
\paragraph{•}
\centerline{LECTURER: MR.ERNEST MWEBAZE}
\paragraph{•}
         \centerline{A LITERATURE REVIEW OF GOOGLE PAY}
          \author{BAMWIREKU PRISCA}
 \paragraph{•}
\centerline{STUDENT NUMBER : 216020499}\
\paragraph{•}
\centerline{REGISTRATION NUMBER:16/U/18716}
\paragraph{•}
%\maketitle
\end{titlepage}

\tableofcontents
\newpage
\pagenumbering{arabic}
\section{introduction.}
Google Pay (formerly Pay with Google and Android Pay) is a digital wallet platform and online payment system developed by Google to power in-app and tap-to-pay purchases on mobile devices, enabling users to make payments with Android phones, tablets or watches.
As of January 8, 2018, the old Android Pay and Google Wallet have unified into a single pay system called Google Pay.[1] Android Pay was rebranded and renamed as Google Pay. It also took over the branding of Google Chrome's autofill feature.[2] Google Pay would have all the features of Android Pay, while Google Wallet features such as requesting and sending money appear in Google Pay Send, currently a separate app.[3][4]
The rebranded service provided a new API that allows merchants to add the payment service to websites, apps, Stripe, Braintree, and Google Assistant.[5] The service allows users to use the payment cards they have on file with Google Play.[6]


\section{BODY.}

\section{CONCLUSION.}

\bibliographystyle{IEEEtran}
\bibliography{refrences}
\end{document}