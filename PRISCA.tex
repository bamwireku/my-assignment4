\documentclass[10pt,a4paper]{article}
\usepackage{cite}
\usepackage{zed-csp,graphicx,color}
\begin{document}
\begin{titlepage}
 \begin{figure}[h]
  \centerline{\small MAKERERE 
  \includegraphics[width=0.1\textwidth]{muklog} UNIVERSITY}
\end{figure}
\centerline{COLLEGE OF COMPUTING AND INFORMATIC SCIENCES}
\paragraph{•}
\centerline{DEPARTMENT OF COMPUTER SCIENCE\\}
\paragraph{•}

\centerline{COURSEWORK THREE: RESEARCH METHODOLOGY(BIT 2207)\\}
\paragraph{•}
\centerline{LECTURER: MR.ERNEST MWEBAZE}
\paragraph{•}
         \centerline{A LITERATURE REVIEW OF GOOGLE PAY}
          \author{BAMWIREKU PRISCA}
 \paragraph{•}
\centerline{STUDENT NUMBER : 216020499}\
\paragraph{•}
\centerline{REGISTRATION NUMBER:16/U/18716}
\paragraph{•}
%\maketitle
\end{titlepage}

\tableofcontents
\newpage
\pagenumbering{arabic}
\section{introduction.}
Google Pay (formerly Pay with Google and Android Pay) is a digital wallet platform and online payment system developed by Google to power in-app and tap-to-pay purchases on mobile devices, enabling users to make payments with Android phones, tablets or watches. As of January 8, 2018, the old Android Pay and Google Wallet have unified into a single pay system called Google Pay. Android Pay was rebranded and renamed as Google Pay. It also took over the branding of Google Chrome's autofill feature. Google Pay would have all the features of Android Pay, while Google Wallet features such as requesting and sending money appear in Google Pay Send, currently a separate app.
The rebranded service provided a new API that allows merchants to add the payment service to websites, apps, Stripe, Braintree, and Google Assistant. The service allows users to use the payment cards they have on file with Google Play.


\section{BODY.}
Google Pay uses near field communication (NFC) to transmit card information facilitating funds transfer to the retailer. It replaces the credit or debit card chip and PIN or magnetic stripe transaction at point-of-sale terminals by allowing the user to upload these in the Google Pay wallet. It is similar to contactless payments already used in many countries, with the addition of two-factor authentication. The service lets Android devices wirelessly communicate with point of sale systems using a near field communication (NFC) antenna, host-based card emulation (HCE), and Android's security.
Users can add payment cards to the service by taking a photo of the card, or by entering the card information manually. To pay at points of sale, users hold their authenticated device to the point of sale system. The service has smart-authentication, allowing the system to detect when the device is considered secure (for instance if unlocked in the last five minutes) and challenge if necessary for unlock information. Spring CEO Alan Tisch said Google Pay improves mobile shopping business by supporting a "buy button" powered by Google Pay integrated within vendor's creative design.

\section{CONCLUSION.}

\bibliographystyle{IEEEtran}
\bibliography{refrences}
\end{document}
